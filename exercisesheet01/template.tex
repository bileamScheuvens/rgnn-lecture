\documentclass[11pt]{article}
\usepackage[utf8]{inputenc}

\usepackage{amssymb, amsmath, amsthm, amsfonts, algorithmic, algorithm, graphicx}
\usepackage{color}
\usepackage{bbm}
\usepackage[dvipsnames]{xcolor} 
\usepackage[colorlinks,linkcolor=blue,citecolor=blue]{hyperref}
\usepackage{array}
\usepackage{ifthen}

\renewcommand{\baselinestretch}{1.1}
\setlength{\topmargin}{-3pc}
\setlength{\textheight}{8.5in}
\setlength{\oddsidemargin}{0pc}
\setlength{\evensidemargin}{0pc}
\setlength{\textwidth}{6.5in}

\newtheorem{theorem}{Theorem}[section]
\newtheorem{lemma}[theorem]{Lemma}
\newtheorem{proposition}[theorem]{Proposition}
\newtheorem{corollary}[theorem]{Corollary}
\newtheorem{question}[theorem]{Question}
\newtheorem{result}[theorem]{Result}
\newtheorem{definition}[theorem]{Definition}
\newtheorem{example}[theorem]{Example}
\newtheorem{remark}[theorem]{Remark}
\newtheorem{assumption}[theorem]{Assumption}
\numberwithin{equation}{section}

\def \endprf{\hfill {\vrule height6pt width6pt depth0pt}\medskip}
\renewenvironment{proof}{\noindent {\bf Proof} }{\endprf\par}

% Notational convenience,
% real numbers 
\newcommand{\R}{\mathbb{R}}  
% Expectation operator
\DeclareMathOperator*{\E}{\mathbb{E}}
% Probability operator
\DeclareMathOperator*{\Prob}{\mathbb{P}}
\renewcommand{\Pr}{\Prob}

% You may define additional macros here.


\begin{document}

\begin{center}
    \sc Recurrent and Generative ANNs
\end{center}

\noindent Name: Bileam Scheuvens, Pankaj Bora

\noindent Email: benedictbileam@gmx.de, bora.pankajt1@gmail.com

\noindent Matr. Nr.:6983475, 6946375



\section*{Exercise 1}
\subsection*{(a)}
See code.
\subsection*{(b)}
\subsubsection*{Why is CE typically paired with Softmax?}
Because cross entropy expects logits, i.e. scores between 0 and 1, which the softmax activation function produces.

\subsubsection{Gradient for CE with Softmax}

\section*{Exercise 2}
See code.

\section*{Exercise 3}

\begin{enumerate}
  \item{
\textbf{Network Configuration}
A hidden size of 3 leads to a hexagonal decision boundary.
The model thus has 3 layers, with 3 neurons each and a ReLU activation function after the first and second hidden layer.
}
\item{
\textbf{Theoretical Explanation}
Each ReLU is essentially a linear boundary.
Since the model has 6 neurons with ReLU and a hexagon is the best available approximation of a circle, it quickly finds this minimum.
  }
\end{enumerate}
\end{document}
